\documentclass[12pt]{article}
\usepackage{algos-tasks}
\usepackage{listings}
\usepackage[table]{xcolor}
\usepackage{array}
\usepackage{xcolor} % pretty colours
\usepackage{varwidth}
\title{2.6 Caffeine}
\author{Andre Monteiro}
\date{Term 3, 2025}

\begin{document}
\maketitle
\newpage
\subsection*{Problem Summary}

A cafe offers $n$ drinks, where the $i$-th drink contains $c_i$ milligrams of caffeine per serving. Buzz has $n$ vouchers, each with value $v_i$, which can be used to redeem $v_i$ servings of a single, unique drink (no two vouchers can be used on the same drink).

The goal is to assign each voucher to a different drink in such a way that the \textbf{total caffeine consumed} (i.e., $\sum v_i \cdot c_j$ for some assignment of drinks $j$) is maximized.


For example, suppose the drinks (say cappuccino, latte, mocha and espresso) have caffeine contents of $600, 200, 400$, and $100$ milligrams and Buzz's vouchers have values of $6, 3, 8$ and $6$.
\begin{center}
\begin{tabular}{|c|c|c|c|c|}
    \hline 
    \(i\) & 1 & 2 & 3 & 4  \\ \hline 
    name & cappuccino & latte & mocha & espresso \\ \hline
    \(c_i\) & 600 & 200 & 400 & 100 \\ \hline
    \(v_i\) & 6 & 3 & 8 & 6 \\ \hline
\end{tabular}
\end{center}
One possibility is to use voucher 1 for six mochas, voucher 2 for three cappuccinos, voucher 3 for eight espressos and voucher 4 for six lattes. The total caffeine content would be 
\begin{align*}
    v_1c_3 + v_2c_1 + v_3c_2 + v_4c_4 &= (6\times400)+(3\times600)+(8\times200)+(6\times100) \\
    &= 6400\ \text{milligrams}.
\end{align*}
    Note that this is \textit{not} the optimal allocation!
\newpage 

\subsection*{Question A)} 

We are given the task of proving that 
\begin{itemize}
    \item We should sort the drinks in ascending order 
    \item We should sort the vouchers in ascending order 
\end{itemize}    
Intuitively this makes sense, but to prove it we need to show that our ordering $G$ is as good or better than any alternative ordering $A$.

Firstly we define the cost function, and as we want to sort by 
\begin{enumerate}
    \item Drinks with \textit{highest} caffeine contents 
    \item Vouchers with \textit{highest} highest redemptions 
\end{enumerate}

If the drinks are ordered $c_1, c_2, \ldots, c_3$, and drink vouchers $v_1, v_2, \ldots, v_3$, then the expected cost function is: 

\[
E = c_1 v_1 + c_2 v_2 + c_3 v_3 + \cdots + c_j v_i
\]

\subsection*{Claim}
The maximized value is achieved when both drinks and vouchers are sorted in ascending order, e.g. lowest cost drink to lowest voucher value

\[
C \in c_1 \le c_2 \le c_3 \le \cdots \le c_j
\]
\[
V \in v_1 \le v_2 \le v_3 \le \cdots \le v_i
\]

\subsection*{Proof by Contradiction}

Suppose there exists ordering $A$, wherein $C$ nor $V$ is in ascending order, and our pairing $G$ where it is. 

We assume $A$ to be maximised, that is, to have the highest possible net value. For the sake of contradiction, there is pairing: 

\[
c_i > c_{i+1}, \quad v_i < v_{i+1}
\]
In this scenario, we have total value:

\[
T = c_i v_i + c_{i+1} v_{i+1}
\]

If we were to switch the pairing, our resultant sum is:

\[
T' = c_i v_{i+1} + c_{i+1} v_i
\]

Now,
\[
T - T' = c_i v_i + c_{i+1} v_{i+1} - (c_i v_{i+1} + c_{i+1} v_i)
\]

\[
= (c_i - c_{i+1}) v_i + (c_{i+1} - c_i) v_{i+1}
\]

\[
= (c_i - c_{i+1})(v_i - v_{i+1})
\]

Since $c_i > c_{i+1}$ and $v_i < v_{i+1}$, this product is negative:

This means that any pairing that is not ordered identically, 
\[
T - T' \le 0 
\]

\[
\boxed{T \le T'} 
\]

Therefore implying a contradiction, as T isn't the maximized sum.  

\subsection*{Question b)}

By the above argument, any inversion in the order (i.e., pairing a higher caffeine drink with a lower-value voucher or vice versa) can be swapped to increase the total caffeine.

Therefore, if we continue eliminating all such inversions, we will eventually arrive at a configuration where no such beneficial swap exists, which is precisely when both sequences are sorted in the same order.

Hence, the greedy matching $G$, where both vouchers and drinks are sorted in ascending order, yields the maximum total caffeine and is thus:

\[
\boxed{\text{Optimal}}
\]
\end{document}