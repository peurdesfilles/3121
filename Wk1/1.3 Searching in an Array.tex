\documentclass[12pt]{article}
\usepackage{algos-tasks}

\begin{document}
\task[regular]{Searching in an Array}

\begin{question}
A two-dimensional array is \emph{sorted} if each of its rows and columns is strictly increasing. For example, the following two-dimensional array is sorted.
\[\begin{bmatrix}
    2 & 3 & 6 & 8 \\
    4 & 5 & 7 & 10 \\
    6 & 8 & 9 & 13 \\
    9 & 11 & 12 & 15
\end{bmatrix}\]

Given a sorted two-dimensional array $A$ with $n$ rows and $n$ columns and an integer $k$, our task is to design an $O(n)$ algorithm that determines whether or not $k$ appears somewhere in $A$.

\begin{enumerate}
    \item Suppose we query the top right entry $A[1][n]$ of the array. In each of the following three cases, explain which parts of the array $A$ may contain $k$, and which parts of the array $A$ definitely cannot contain $k$.
    \begin{itemize}
        \item {\bfseries Case 1}: $A[1][n] > k$.
        \item {\bfseries Case 2}: $A[1][n] < k$.
        \item {\bfseries Case 3}: $A[1][n] = k$.
    \end{itemize}

    In each case, briefly justify your answer.

    \item \label{algorithm_part} Design an $O(n)$ algorithm that determines whether or not $k$ appears in $A$.

    {\bfseries Hint.} \emph{After making a query to $A[1][n]$, what are the dimensions of the search space? How could you use that information to construct the algorithm?}

    \item \textbf{(Optional)} If we instead began by examining the entry in the middle row and middle column\footnote{or one of the middle rows and columns, if the dimensions are even.}, what shape does the remaining search space have? How does this make searching in a two-dimensional array more difficult than the proposed algorithm in part \ref{algorithm_part}?
\end{enumerate}
\end{question}

\newpage

\begin{rubric}

\begin{itemize}

 \item\textbf{Clarity:} For part (a), consider each case separately, and:
    \begin{itemize}
        \item Provide a brief but clear, and correct description of which part(s) of the original array $k$ can appear in. You don't have to give too strong a claim.
        \item Provide one to two sentences of justification for your answer to the previous dot point.
    \end{itemize}

    \item {\bfseries (Optional) For part (c):} It is sufficient to describe the shape of the remaining search space using the proposed querying algorithm. Provide a brief but clear argument for why searching for $k$ is more difficult than with the algorithm from part \ref{algorithm_part}.

\end{itemize}
 \expected[3]{two to three sentences per case., half a page., two to three sentences.}
\end{rubric}
\newpage
\begin{solution}
% Write your solution here.
\end{solution}
\begin{attribution}
% Write any attributions here.
\end{attribution}

\end{document}