\documentclass[12pt]{article}
\usepackage{algos-tasks}

\begin{document}
\task[regular]{Geese in a Trenchcoat II}

\begin{question}
Recall the example problem \emph{Geese in a Trenchcoat} from the first lecture.

Consider the following alternative algorithm for the same problem, which uses a ``two pointers'' method.

\begin{quote}
    Sort the geese from left to right in ascending order of height, using merge sort. Put a red hat on the shortest goose and a blue hat on the tallest goose. Then:
    \begin{itemize}
        \item if the hatted geese have total height less than $H$, move the red hat one space to the right;
        \item if the hatted geese have total height greater than $H$, move the blue hat one space to the left, and
        \item if the hatted geese have total height equal to $H$, report this pair as a solution.
    \end{itemize}
    If a single goose ever has both hats, report that there is no solution.
\end{quote}

Justify why the algorithm is correct, and analyse its time complexity.

\hint When we remove a hat (say the red hat) from a particular goose, what are we saying about that goose, and how do we know that conclusion to be true?
\end{question}
\begin{rubric}
\begin{itemize}
    \item Explain why, if there is a solution, then it is found by this algorithm.
    \begin{itemize}
        \item You'll need to justify that each move of the hats is valid, i.e. doesn't lose any solutions.
    \end{itemize}
    \item Explain why, if there is no solution, then the algorithm reports that there is no solution. This part should be easy!
    \item To analyse the time complexity, consider the sorting as well as the part with the hats.
    \begin{itemize}
        \item How many times can each hat move?
    \end{itemize}
\end{itemize}
 \expected[1]{half a page.}
\end{rubric}
\clearpage
\begin{solution}
% Write your solution here.
\end{solution}
\begin{attribution}
% Write any attributions here.
\end{attribution}

\end{document}