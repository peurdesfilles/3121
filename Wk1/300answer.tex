\documentclass[12pt]{article}
\usepackage{algos-tasks}
\usepackage{listings}
\usepackage[table]{xcolor}
\usepackage{array}
\usepackage{xcolor} % pretty colours
\usepackage{varwidth}
\title{1.4 300}
\author{Andre Monteiro}
\date{Term 3, 2025}

\begin{document}
\maketitle
\newpage
\section*{Problem Summary}

We have a classroom of 300 students arranged in a \(15 \times 20\) grid.  
Define:
\begin{itemize}
    \item \textcolor{green}{A} = tallest among the shortest students in each column
    \item \textcolor{red}{B} = shortest among the tallest students in each row
\end{itemize}

We aim to compare \(A\) and \(B\), and illustrate the pattern using smaller sample matrices.

\subsection*{Sample Matrix 1}


\[
\begin{minipage}{0.45\textwidth}
\[ A =
\begin{bmatrix}
2 & 5 & 3 \\
4 & \textcolor{green}{1} & 6 \\
\textcolor{green}{1} & 3 & \textcolor{green}{2}
\end{bmatrix}
\]
\end{minipage}
\hfill
\begin{minipage}{0.45\textwidth}
\[ B =
\begin{bmatrix}
2 & \textcolor{red}{5} & 3 \\
\textcolor{red}{4} & 1 & 6 \\
1 & \textcolor{red}{3} & 2
\end{bmatrix}
\]
\end{minipage}
\]

\paragraph{Solution}
\begin{itemize}
    \item In the first matrix, \(A = 2\). 
    \item In the second matrix, \(B = 3\).  
\end{itemize}

\[
\boxed{A < B}
\]

\subsection*{Sample Matrix 2}


\[
\begin{minipage}{0.45\textwidth}
\[ A =
\begin{bmatrix}
7 & \textcolor{green}{4} & 5 \\
3 & 6 & \textcolor{green}{2} \\
\textcolor{green}{1} & 5 & 6
\end{bmatrix}
\]
\end{minipage}
\hfill
\begin{minipage}{0.45\textwidth}
\[ B =
\begin{bmatrix}
\textcolor{red}{7} & 4 & 5 \\
3 & \textcolor{red}{6} & 2 \\
1 & 5 & \textcolor{red}{6}
\end{bmatrix}
\]
\end{minipage}
\]

\paragraph{Solution}

\begin{itemize}
    \item In the first matrix, \(A = 4\).
    \item In the second matrix, \(B = 6\).
\end{itemize}

\[
\boxed{A < B}
\]
\newpage

\subsection*{Proof}

\begin{itemize}
    \item For each column, we found the shortest student. Then \(A\) is the tallest among these.  
      So \(A\) is bigger than or equal to any of the column minima.
    \item For each row, we found the tallest student. Then \(B\) is the shortest among these.  
      So \(B\) is smaller than or equal to any of the row maxima.
\end{itemize}

Now, look at any student in the grid. Each student is **at least as tall as the shortest in their column** and **at most as tall as the tallest in their row**.  

This means the tallest of all column minima (\(A\)) can never be taller than the shortest of all row maxima (\(B\)), and since we also apply the restriction:
\[
\boxed{A \ne B}
\]
We can therefore state that no matter how the students are arranged:

\[
\boxed{A < B, (A \ne B)}
\]

The previous examples make this clear in practice: green (A) is always less than red (B).

\end{document}