\documentclass[12pt]{article}
\usepackage{algos-tasks}
\usetikzlibrary{shapes}
\usepackage{subcaption}
\usepackage{soul}

\colorlet{myred}{red!60!black}
\colorlet{myblue}{blue!60!black}

\newcommand{\R}[1]{{\color{myred}#1}}
\newcommand{\B}[1]{{\color{myblue}#1}}

\DeclareMathOperator{\arr}{\mathsf{Arrive}}
\DeclareMathOperator{\dep}{\mathsf{Depart}}
\DeclareMathOperator{\toll}{\mathsf{Toll}}
\DeclareMathOperator{\s}{\mathsf{Start}}
\DeclareMathOperator{\e}{\mathsf{End}}

\begin{document}

\task[regular]{Road Trip}
\begin{question}
You are going on a road trip to Austria, but are on a limited budget. Austria has $n$ towns, each with an airport. Each town $t$ has a cost of $\arr(t)$ to enter Austria via its associated airport and a cost of $\dep(t)$ to leave from its airport.

There are also $m$ roads between the towns.
\begin{itemize}
  \item Each road $r$ has an associated toll cost $\toll(r)$.
  \item It is possible to reach any town from any other town by a sequence of roads. 
\end{itemize}
All arrival, departure and toll costs are non-negative.

A \textit{road trip} is a path through Austria where you arrive at a town by plane, visit one or more towns via roads, and then leave by plane (possibly from the town where you started). In other words, a road trip must involve travelling over at least one road -- flying to a town then immediately flying out without driving anywhere is not a road trip! You want to determine the cost of the cheapest road trip you can take.

\begin{figure}[H]
    \centering
    \begin{tikzpicture}[{every text node part/.style={align=center}}, scale = 0.8]
      \node[draw, circle] (a) at (0,0) {Vienna \\ \R{$\mathsf{A} = 2$} \\ \B{$\mathsf{D} = 8$}};
      \node[draw, circle] (b) at (6,0) {Graz \\ \R{$\mathsf{A} = 5$} \\ \B{$\mathsf{D} = 5$}};
      \node[draw, circle] (c) at (3,2) {Salzburg \\ \R{$\mathsf{A} = 10$} \\ \B{$\mathsf{D} = 1$}};
      \node[draw, circle] (d) at (10,2) {Innsbruck \\ \R{$\mathsf{A} = 1$} \\ \B{$\mathsf{D} = 1$}};
    
      \path (a) edge node [below] {1} (b);
      \path (a) edge node [above left] {5} (c);
      \path (c) edge node [above right] {2} (b);
      \path (b) edge node [above left] {100} (d);
    \end{tikzpicture}
    \caption{}
    \label{example-1}
\end{figure}

In Figure \ref{example-1}, the cheapest road trip is to enter at Vienna, drive to Graz then Salzburg, and leave via Salzburg. The total cost is $$\arr(\text{Vienna}) + \toll(\text{Vienna}, \text{Graz}) + \toll(\text{Graz}, \text{Salzburg}) + \dep(\text{Salzburg}) = 2 + 1 + 2 + 1 = 6.$$ While it would be cheaper to fly into Innsbruck then immediately fly out, this is not a road trip as no roads are taken. Finally, a road trip does not have to include all towns.

\newpage

\begin{enumerate}[label = (\alph*)]
    \item By constructing a new graph such that each path in the new graph corresponds to a valid road trip, design an algorithm to find the cost of the cheapest road trip in $O(m\log n)$ time.
    
    {\bfseries Hint.} {\em How do you account for the requirement that at least one road must be used? It might help to draw out a few examples.}
    
    \item \textbf{(Optional)} Suppose we are required to make a \emph{long} road trip, defined as a road trip using \emph{at least three} (not necessarily distinct) roads. Design an efficient algorithm that finds the cheapest long road trip, \emph{using no more than $2n+2$ vertices in your graph construction}.
  
\end{enumerate}
{\bfseries Note:} We are again using the method of \textit{problem solving by reduction}, that is, we are taking the problem we are interested in solving and transforming it into a problem we already know how to solve. This applies to both part (a) and part (b).


\end{question}

\begin{rubric}

 \expected[3]{half a page, half a page.}


\end{rubric}

\newpage

\begin{solution}
% Write your solution here.
\end{solution}
\begin{attribution}
% Write any attributions here.
\end{attribution}
\end{document}