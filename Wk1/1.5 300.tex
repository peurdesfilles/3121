\documentclass[12pt]{article}
\usepackage{algos-tasks, subcaption}

\begin{document}
\task[regular]{300}

\begin{question}
There are 300 students in a classroom. The students are organised into a grid of 15 rows and 20 columns. We select two students as follows. \begin{itemize}
    \item From each column of students, we find the shortest student. We then select the tallest of these students (i.e. the tallest of the shortest). This student is  $A$.
    \item From each row of students, we find the tallest student. We then select the shortest of these students (i.e. the shortest of the tallest). This student is $B$.
\end{itemize}

Suppose that $A$ is \emph{not} the same height as $B$. Is it true that in some cases $A$ is taller than $B$, and in other cases, $B$ is taller than $A$? Provide a proof of your conclusion. 

{\bfseries Hint.} {\em Investigate several smaller examples of the problem to develop a conjecture for your conclusion, and then prove the conjecture generally.}
\end{question}
\begin{rubric}

\begin{itemize}
    \item \clarity You'll need to introduce some mathematical notation for the heights of the students occupying each position in the grid. You should use this notation consistently and clearly throughout your answer.
\end{itemize}
    \expected[1]{half a page.}
\end{rubric}
\clearpage
\begin{solution}
% Write your solution here.
\end{solution}
\begin{attribution}
% Write any attributions here.
\end{attribution}
\end{document}
